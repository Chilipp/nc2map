% !TeX root = ../user_manual.tex
\chapter{Evaluation routines} \label{ch:eval}
There are three possible evaluation methods that are incorporated in the \gls{nc2map} module. I will explain the main principles for application, however please look into the \lstinline|nc2map/demo| directory for direct application examples and use the python \lstinline|help| function.

\section{Incorporation of Climate Data Operators} \label{sec:cdo}
\glspl{cdo} are implemented via the \glssymbol{cdo} class, which itself is based upon the \lstinline|Cdo| class of the \lstinline|cdo.py| python module\footnote{\url{https://code.zmaw.de/projects/cdo/wiki/Cdo{rbpy}}}. There are four new keywords implemented for each operator:
\begin{description}
	\item[returnMaps] takes None, a string or list of strings (the variable
	              names) or a dictionary (keys and values are determined by the
	              \gls{Maps} method). None will open maps for all
	              variables that have longitude and latitude dimensions in it.
	              An open \gls{Maps} instance is returned.
	\item[returnMap] takes a string (the variable name) or a dictionary
	              (keys and values are determined by the
	              \gls{FieldPlot} method). An open \gls{FieldPlot}
	              instance is returned
	\item[returnLine] takes a string or list of strings (the variable
	              names) or a dictionary (keys and values are determined by the
	              \gls{LinePlot} method). An open \gls{LinePlot} instance is returned
	\item[returnData] takes a string or list of strings (the variable
	              names) or a dictionary (keys and values are determined by
	              the \gls{reader.get_data} method).
	              It returns a \gls{DataField} instance of the
	              specified variables, with datashape='any'.
\end{description}
Hence you can not only evaluate your results with cdos, but also immediately visualize the data. See the \lstinline|nc2map/demo/cdo_demo.py| for a demonstration of the possibilities.

\section{Calculation} \label{sec:calculation}
As stated in \autoref{ch:data}, there exist basically three levels. On the lowest two levels (the \gls{MapBase} and \gls{reader} level), you can perform calculations like multiplication, subtraction, power, division or addition. The great advantage is, that you immediately can visualize your result, change meta data, etc. You can find a demo file in \lstinline|nc2map/demo/calculation_demo.py|.

There are however a few rules that you should consider concerning arithmetics between \gls{MapBase} instances:
\begin{enumerate}
	\item When you apply arithmetics with MapBase instances, the MapBase first extracts it's data in it's \gls{reader} and creates a new reader where it now takes the data from. Therefore, the new reader will (for example) have only one single time step, one single level, etc..
	\item You can add floats, \lstinline|numpy.ndarrays| matching the shape of the \gls{MapBase}\lstinline|.data| attribute, other \gls{MapBase} instances\footnote{you can only calculate between scalar fields (i.e. \gls{FieldPlot} with \gls{FieldPlot}) or between vector fields (i.e. \gls{WindPlot} with \gls{WindPlot}), but not mix the two classes} or other \glspl{reader}.
	\item The resulting \gls{MapBase} will have set all dimensions set to 0 (this does only matter, if you perform arithmetics with a \gls{reader}).
\end{enumerate}

If you want to perform arithmetics between readers (e.g. to consider the full data and not only one two-dimensional array), there are also some rules:
\begin{enumerate}
	\item You can add floats, \lstinline|numpy.ndarrays| matching the shape of the \gls{MapBase}\lstinline|.data| attribute or other \glspl{reader}. If you have a \lstinline|numpy.ndarrays|, the shape has to match the shape of the variables in the reader (this implies that all variables that are not dimensional data (e.g. \gls{time}) must have the same shape).
	\item If you add another \gls{reader}, it can have 
	\begin{enumerate}
		\item all the same variables or only one variable which will then be added to the variables of the other \gls{reader}
		\item only one time step which will then be added to all the other time steps in the other \gls{reader}
		\item only one level which will then be added to all the other levels in the other \gls{reader}
	\end{enumerate}
\end{enumerate}
It has not been evaluated so far with large data sets, but feel free to do it and I would be happy for results :) However it is generally faster to make calculations on the \gls{reader} level than on the \gls{MapBase} level.

\section{Evaluator classes} \label{sec:evaluators}
Additionally to the cdo interface (see \autoref{sec:cdo}) there are (currently) two evaluators implemented (the \gls{FldMeanEvaluator} and \gls{ViolinEvaluator}), which you can access via the \gls{Maps.evaluate} method of your \gls{Maps} instance. Those evaluators both possibly can evaluate multiple regions at the same time. The region definition is thereby determined by the \hyperref[item:mask]{mask} \gls{fmt} keyword. However you can of course simply zoom to the region that you are interested in (see the \hyperref[item:lonlatbox]{lonlatbox} keyword) and make an evaluation without specifying the regions.

For an example look into the \lstinline|nc2map/demo/evaluators_demo.py| script.