% !TeX root = ../user_manual.tex
\chapter{Introduction}
Visualizing data is a major part of the scientific work, not only for publications but also to analyse ones own data. However commonly large scripts or more or less complex programmes are necessary to visualize your data especially when it comes to two-dimensional global maps. Therefore this module has been developed with a special focus on the visualization of NetCDF files, a commonly used data format in climate sciences (at least when it comes to global models), to efficiently visualize the data. They major advantages compared to other programmes and modules are:
\begin{enumerate}
	\item with only a small number of commands you can take a detailed look into your data and make nicely looking maps (for publications or just to get an idea of your data)
	\item you can easily access the data, make calculations with it and implement everything into your own plotting and evaluation routines since there are very weak dependencies
	\item it is (hopefully) well documented
	\item it is open source
\end{enumerate}
The final goal is to also develop a graphical user interface (GUI) for the module to create something like \lstinline|ncview| just better. However, since the module is very new, there is currently only support for the command line application, i.e. for the use in scripts or with \lstinline|python|, \lstinline|ipython|, etc.

