% MapBase
\newglossaryentry{MapBase}{name=MapBase, description={Basic class that reads data from a NetCDF file and plots visualizes it on one axes}, type=main, text=\lstinline|MapBase|, symbol=\lstinline|nc2map.mapos.MapBase|}
\newglossaryentry{FieldPlot}{name=FieldPlot, description={Class that plots a two-dimensional field (e.g. temperature) and optionally overlained by a wind field}, type=main, text=\lstinline|FieldPlot|, parent=MapBase, symbol=\lstinline|nc2map.mapos.FieldPlot|}
\newglossaryentry{FieldPlot.data}{name=data, description={\gls{DataField} instance storing the data of the variable}, text=\lstinline|data|, symbol=\lstinline|FieldPlot.data|, parent=FieldPlot}
\newglossaryentry{WindPlot}{name=WindPlot, description={Class that plots a wind field (stream plot or quiver plot)}, type=main, text=\lstinline|WindPlot|, parent=MapBase, symbol=\lstinline|nc2map.mapos.WindPlot|}
\newglossaryentry{MapBase.meta}{name=meta, description={meta data property of the \gls{MapBase} instance which gives a dictionary containing all the meta data information of the specific variable}, symbol=\lstinline|MapBase.meta|, text=\lstinline|MapBase.meta|, symbol=\lstinline|MapBase.meta|, parent=MapBase}
\newglossaryentry{LinePlot}{name=LinePlot, description={Class to visualize one dimensional data in the NetCDF file}, text=\lstinline|LinePlot|, symbol=\lstinline|nc2map.mapos.LinePlot|}
\newglossaryentry{ViolinPlot}{name=ViolinPlot, description={Class to make a violin plot using the python \lstinline|seaborn.violinplot| function. Note that if you want to create an \lstinline|ViolinPlot| instance manually, this class (different from \gls{LinePlot} and \gls{MapBase} instances) does not extract the data from a \gls{reader}. Instead you have to pass it in manually at the initialization}, text=\lstinline|ViolinPlot|, symbol=\lstinline|nc2map.mapos.ViolinPlot|, see=ViolinEvaluator}
\newglossaryentry{MapBase.get_data}{name=MapBase.get\_data, description={Method of \gls{MapBase} (and \gls{LinePlot}) instances to extract the data from the reader}, parent=MapBase, see=reader.get_data, symbol=\lstinline|nc2map.mapos.MapBase.get_data|, text=\lstinline|get_data|}

% nc2map
\newglossaryentry{nc2map}{name=nc2map, description={Interactive python module to visualize NetCDF files on a map}, type=main, text=\lstinline|nc2map|}
\newglossaryentry{nc2map.load}{name=load, description={Function that loads \lstinline|pickle| files generated by the \gls{Maps.save} method and opens the \gls{Maps} instance with the previously saved settings}, text=\lstinline|load|, symbol={\lstinline|nc2map.load|}, parent=nc2map, see={Maps.save}}

\newglossaryentry{shp_utils}{name={shp\_utils}, description={Shape file module of the \gls{nc2map} module}, text=\lstinline|shp_utils|, symbol={\lstinline|nc2map.shp_utils|}}

% MapsManager
\newglossaryentry{MapsManager}{name=MapsManager, description={Basic class that controls multiple \gls{MapBase} instances}, type=main, text=\lstinline|MapsManager|, symbol=\lstinline|nc2map.MapsManager|}
\newglossaryentry{Maps}{name=Maps, description={Basic class for plotting in \gls{nc2map}. Controls multiple \gls{MapBase} instances}, type=main, text=\lstinline|Maps|, parent=MapsManager, symbol=\lstinline|nc2map.Maps|}
\newglossaryentry{Maps.update}{name=update, description={Update method of the \gls{Maps} class. Updates the chosen \gls{MapBase} instances by the given formatoptions. \gls{MapBase} instances may be chosen via \glspl{dimsidentifier}}, type=main, text=\lstinline|update|, parent=Maps, symbol={\lstinline|Maps.update|}}
\newglossaryentry{Maps.output}{name=output, description={Output method of the \gls{Maps} class. Exports the chosen \gls{MapBase} instances (or figures) into different (static) formats. \gls{MapBase} instances may be chosen via \glspl{dimsidentifier}}, type=main, text=\lstinline|output|, parent=Maps, symbol={\lstinline|Maps.output|}, see={Maps.make_movie}}
\newglossaryentry{Maps.make_movie}{name={make\_movie}, description={Movie method of the \gls{Maps} class. Exports the chosen \gls{MapBase} instances (or figures) into a movie of the specified format for the user given (or all) time steps. \gls{MapBase} instances may be chosen via \glspl{dimsidentifier}}, type=main, text=\lstinline|make_movie|, parent=Maps, symbol={\lstinline|Maps.make_movie|}, see={Maps.output}}
\newglossaryentry{MapsManager.addmap}{name={addmap}, description={Add a new MapBase instance to the \gls{MapsManager} instance. This function is used at the initialization of a \gls{Maps} instance.}, parent=MapsManager, text=\lstinline|addmap|, symbol=\lstinline|MapsManager.addmap|}
\newglossaryentry{MapsManager.addline}{name={addline}, description={Add a new LinePlot instance to the \gls{MapsManager} instance. This function is used at the initialization of a \gls{Maps} instance if \lstinline|linesonly=True| is set.}, parent=MapsManager, text=\lstinline|addline|, symbol=\lstinline|MapsManager.addline|}
\newglossaryentry{MapsManager.get_label_dict}{name={get\_label\_dict}, description={Helper function. Returns the meta data of the given \gls{MapBase} instance}, text=\lstinline|get_label_dict|, parent=MapsManager, symbol={\lstinline|Maps.get_label_dict|}, see={MapsManager.meta,MapBase.meta}}
\newglossaryentry{MapsManager.meta}{name={meta}, description={Property that returns a dictionary containing all the meta information in the \gls{MapBase} and \gls{LinePlot} instances of the \gls{MapsManager} instance}, parent=MapsManager, text=\lstinline|MapsManager.meta| see={MapsManager.get_label_dict,MapBase.meta}}
\newglossaryentry{MapsManager.get_maps}{name={get\_maps}, description={Helper function. Returns the \gls{MapBase} corresponding to the given \gls{dimsidentifier}}, text=\lstinline|get_maps|, parent=MapsManager, symbol={\lstinline|MapsManager.get_maps|}}
\newglossaryentry{Maps.save}{name=save, description={Helper function of the \gls{Maps} class. This method creates a \lstinline|pickle| file that can be loaded with the \gls{nc2map.load} function, to reinitialize of \gls{Maps} instance with all coordinated \gls{MapBase} instances and their settings}, text=\lstinline|save|, parent=Maps, symbol=\lstinline|Maps.save|, see={nc2map.load}}
\newglossaryentry{Maps.evaluate}{name=evaluate, description={Evaluator method that passes \gls{MapBase} instances to evaluators}, text=\lstinline|evaluate|, symbol=\lstinline|nc2map.Maps.evaluate|, parent=Maps, see={FldMeanEvaluator,ViolinEvaluator}}
\newglossaryentry{Maps.nextt}{name=nextt, description={Updates all MapBase instances controlled by the \gls{Maps} instances to the next time step}, parent=Maps, symbol=\lstinline|Maps.nextt|, see={Maps.prevt}, text=\lstinline|nextt|}
\newglossaryentry{Maps.prevt}{name=prevt, description={Updates all MapBase instances controlled by the \gls{Maps} instances to the next time step}, parent=Maps, symbol=\lstinline|Maps.prevt|, see={Maps.nextt}, text=\lstinline|prevt|}



\newglossaryentry{dimsidentifier}{name={Dimension identifiers}, description=\nopostdesc, type=main, plural={dimension identifiers}, text={dimension identifier}}
\newglossaryentry{var}{name=var, text={\lstinline|var|}, description={(plural: \texttt{vlst}). Identifier for the variable}, type=main, plural={\lstinline|vlst|}, parent=dimsidentifier}
\newglossaryentry{time}{name=time, text={\lstinline|time|}, description={Identifier for the time (as integer)}, type=main, parent=dimsidentifier}
\newglossaryentry{level}{name=level,text={\lstinline|level|}, description={Identifier for the vertical level (as integer)}, type=main, parent=dimsidentifier}
\newglossaryentry{name}{name=name, text={\lstinline|name|}, description={Identifier for the name of the \gls{MapBase} instance}, type=main, parent=dimsidentifier}

\newglossaryentry{fmt}{name=formatoption, symbol={\lstinline|fmt|}, description={Formatoption keywords that control the appearance of the plot (see \autoref{ch:fmt} for details).}}
\newglossaryentry{show_fmtkeys}{name={show\_fmtkeys}, description={Helper function to display the possible \gls{fmt} keywords}, symbol={\lstinline|nc2map.show_fmtkeys|}, text={\lstinline|show_fmtkeys|}, parent=fmt, see=get_fmtkeys}
\newglossaryentry{get_fmtkeys}{name={get\_fmtkeys}, description={Same as \gls{show_fmtkeys}, but returns a string.}, symbol={\lstinline|nc2map.get_fmtkeys|}, text={\lstinline|get_fmtkeys|}, parent=fmt, see=show_fmtkeys}
\newglossaryentry{show_fmtdocs}{name={show\_fmtdocs}, description={Helper function to display the possible \gls{fmt} keywords and their documentation}, symbol={\lstinline|nc2map.show_fmtdocs|}, text={\lstinline|show_fmtdocs|}, parent=fmt, see=get_fmtdocs}
\newglossaryentry{get_fmtdocs}{name={get\_fmtdocs}, description={Same as \gls{show_fmtdocs}, but instead of displaying the documentation, returns a dictionary with formatoption keywords as keys and their documentation as value.}, symbol={\lstinline|nc2map.get_fmtdocs|}, text={\lstinline|get_fmtdocs|}, parent=fmt, see=show_fmtdocs}

\newglossaryentry{show_colormaps}{name={show\_colormaps}, description={Displays all predefined colormaps (or the one specified by the user)}, text={\lstinline|show_colormaps|}, symbol={\lstinline|nc2map.show_colormaps|}}

\newglossaryentry{nc2map.get_fnames}{name={get\_fnames}, description={Displays all possible field names that are in the default shape file used by the \hyperref[item:lineshapes]{lineshapes} formatoption.}, see={nc2map.get_unique_vals}}
\newglossaryentry{nc2map.get_unique_vals}{name={get\_unique\_vals}, description={Displays all possible values that are in the default shape file used by the \hyperref[item:lineshapes]{lineshapes} formatoption.}, see={nc2map.get_fnames}}

% evaluators
\newglossaryentry{evaluator}{name=evaluator, description={Generally a subclass of the \lstinline|EvaluatorBase| class, which is designed to evaluate multiple \gls{MapBase} instances.}, symbol=\lstinline|nc2map.evaluators.EvaluatorBase|, text=\lstinline|evaluator|, plural=\lstinline|evaluators|, see={Maps.evaluate}}
\newglossaryentry{FldMeanEvaluator}{name=FldMeanEvaluator, description={Calculates the weighted 2-dimensional field mean (i.e. the mean over all longitudes and latitudes), saves the data into new \gls{ArrayReader} instances and creates \gls{LinePlot} instances that show the data. You can do that for multiple regions at the same time and include errors. The key in the \gls{Maps.evaluate} method is \lstinline|'fldmean'|}, symbol=\lstinline|nc2map.evaluators.FldMeanEvaluator|, text=\lstinline|FldMeanEvaluator|, parent=evaluator}
\newglossaryentry{ViolinEvaluator}{name=ViolinEvaluator, description={Creates \gls{ViolinPlot} instances that show the data. You can do that for multiple regions at the same time. The key in the \gls{Maps.evaluate} method is \lstinline|'violin'|}, symbol=\lstinline|nc2map.evaluators.ViolinEvaluator|, text=\lstinline|ViolinEvaluator|, parent=evaluator, see=ViolinPlot}

% readers
\newglossaryentry{reader}{name=reader, text=\lstinline|reader|, description={General identifier for a \lstinline|nc2map.readers.ReaderBase| instance}, symbol=\lstinline|nc2map.reader.ReaderBase|, plural=\lstinline|readers|}
\newglossaryentry{ArrayReader}{name=ArrayReader, text=\lstinline|ArrayReader|, description={Base class for the data management in \gls{nc2map}. This class is initialized by the raw data stored in a dictionary}, parent=reader, symbol=\lstinline|nc2map.readers.ArrayReader|}
\newglossaryentry{NCReader}{name=NCReader, text=\lstinline|NCReader|, description={\gls{reader} subclass based upon the \lstinline|netCDF4.Dataset| class to read one single NetCDF file}, parent=reader, symbol=\lstinline|nc2map.readers.NCReader|}
\newglossaryentry{MFNCReader}{name=MFNCReader, text=\lstinline|MFNCReader|, description={\gls{reader} subclass based upon the \lstinline|netCDF4.MFDataset| class to read multiple NetCDF files}, parent=reader, symbol=\lstinline|nc2map.readers.MFNCReader|}
\newglossaryentry{DataField}{name=DataField, text=\lstinline|DataField|, symbol=\lstinline|nc2map.readers.DataField|, description={Basic data class in the \gls{nc2map} module containing the variable data as well as dimension informations (latitude, longitude, time, level, etc.)}, see={reader.get_data}}
\newglossaryentry{reader.get_data}{name={get\_data}, description={\lstinline|get_data| method of the \gls{reader} class to easiliy access the data stored in the \gls{reader} (e.g. \gls{NCReader} or \gls{MFNCReader}) instance}, parent=reader, text=\lstinline|get_data|, symbol=\lstinline|nc2map.reader.ReaderBase.get_data|}

% misc
\newglossaryentry{cdo}{name=CDOs, description={Climate Data Operators. They can be optionally used in combination with the NetCDF package. See \url{https://code.zmaw.de/projects/cdo} for a documentation}, text=\lstinline|cdo|, plural=\lstinline|cdos|, symbol=\lstinline|nc2map.Cdo|}